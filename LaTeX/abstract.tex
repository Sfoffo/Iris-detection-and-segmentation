\begin{comment}
Comandi per eliminare i numeri dall'abstract:

\pagestyle{empty}
\let\originalsecdef\secdef
\def\secdef{\thispagestyle{empty}\originalsecdef}
\thispagestyle{empty}

\end{comment}

\pagestyle{plain}
 
\chapter{Abstract}
Un framework biometrico offre una prova di identità automatica in base a caratteristiche uniche dell'individuo. Inoltre, la comunità scientifica è prevalentemente d'accordo sul fatto che il riconoscimento dell'iride sia il framework di identificazione biometrica più preciso e affidabile disponibile.
Con la crescita delle richieste riguardanti l'identificazione sicura e dato che l'iride umana offre un pattern robusto per l'identificazione, l'utilizzo di strumenti poco costosi potrebbe rendere il riconoscimento dell'iride un nuovo standard nei security framework. È stato sviluppato un sistema di riconoscimento e segmentazione (identificazione dei confini al fine di estrarre solo le informazioni rilevanti) dell'iride sulla base di tecniche biometriche e sulla base di algoritmi, quali Viola-Jones, con lo scopo di mostrare il potenziale offerto dalla segmentazione dell'iride umana a partire da una riproduzione live, utilizzando un dispositivo comune quale la videocamera.
In particolare, il sistema localizza gli occhi per poi estrarne le regioni circolari (pupilla e iride) dalle quali viene ottenuta la segmentazione tramite polarizzazione dell'immagine.