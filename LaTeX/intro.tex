%\pagestyle{empty}
\fancyhf{}
\cfoot{\thepage}
\pagestyle{fancy}
\chapter{Introduzione}

Il lavoro di tesi, svolto durante l’attività di tirocinio presso l'Università degli studi di Salerno nella sede di Fisciano (SA), si colloca nell' ambito della ricerca e sviluppo riguardante l'integrazione di servizi di segmentazione dell'iride umana. In particolare, utilizzando \textit{Haar Cascades} per il riconoscimento del viso e degli occhi e basandosi su algoritmi di \acf{ML} è stato sviluppato un servizio di segmentazione dell'iride utilizzando un dispositivo comune quale la webcam.
Utilizzando tale pattern si potranno sviluppare sistemi di autenticazione sicura.
I sistemi biometrici sono in grado di fornire un livello di
sicurezza più elevato rispetto ad altri sistemi di autenticazione basati su password, ma esistono alcuni problemi legati alle caratteristiche della biometria stessa (alcune cambiano nel tempo in modo significativo) o ai dispositivi utilizzati per catturarle (alcuni possono essere indotti in errore o possono avere difficoltà ad acquisire il tratto biometrico) che scoraggiano la loro diffusione. L'obiettivo del servizio è quello di mostrare il potenziale del riconoscimento e della segmentazione dell'iride. 

Nel capitolo \ref{cap:biometria} vengono introdotti i concetti necessari per la comprensione degli approcci descritti in seguito.

Nel capitolo \ref{cap:CenniAIeML} viene illustrata la letteratura sulle tecniche e sui modelli di Machine Learning e i fondamenti di \acf{AI} cui si basa parte della progettazione del servizio prodotto. Vengono inoltre illustrati accenni sullo stato dell'arte del Machine Learning.

Nel capitolo \ref{cap:NNeCNN} viene illustrato un possibile approccio per lo sviluppo del software prodotto e le motivazioni dietro la possibile scelta di un tale approccio

Nel capitolo \ref{cap:ViolaHaar} viene illustrato l'approccio utilizzato per lo sviluppo del software, confrontandolo all'approccio descritto nel precedente capitolo e motivandone l'utilizzo.

Nel capitolo \ref{cap:conclusioni} vengono presentate le conclusioni e i possibili sviluppi futuri.