\selectlanguage{italian}
\graphicspath{ {img/6/} }
\chapter{Conclusioni e sviluppi futuri}\label{cap:conclusioni}
\thispagestyle{empty}
\newpage

\section{Conclusioni}
\vspace{8mm}
Il software realizzato ha lo scopo di mostrare il potenziale di un framework di riconoscimento e segmentazione dell'iride in contesti non controllati, utilizzando un dispositivo di utilizzo comune.
Attraverso ulteriori sviluppi tecnologici e attraverso dispositivi ad-hoc per il riconoscimento dell'iride è possibile ottenere un sistema in grado di raggiungere le capacità di un framework basato su riconoscimento delle impronte digitali, possibilmente anche superandole.

\section{Sviluppi futuri}
\vspace{8mm}
Il software utilizza una comune webcam di scarsa qualità, sostituendo l'hardware con un dispositivo ad-hoc è possibile raggiungere risultati molto più rilevanti.
Un possibile miglioramento si basa sull'approccio alla CNN, che però necessita di un dataset molto più ampio e strutturato di quelli attualmente esistenti.
